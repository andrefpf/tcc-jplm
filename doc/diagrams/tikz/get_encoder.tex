%!TEX jobname = get_encoder
%!TEX output_directory = ./output/

% @Author: Ismael Seidel
% @Date:   2020-02-05
% @Last Modified by:   Ismael Seidel
% @Last Modified time: 2020-02-07


\documentclass[a4paper,landscape]{article}
\usepackage{pgf,tikz}
\usepackage{verbatim}
\usepackage[active,tightpage]{preview}
\PreviewEnvironment{tikzpicture}
\setlength\PreviewBorder{0pt}%
\usetikzlibrary{calc,arrows,fit, positioning, arrows.meta, fit}
\usetikzlibrary{shapes.gates.logic.US,shapes.geometric,quotes}

\usepackage{pgf}
\usepackage{tikz}
\usetikzlibrary{backgrounds, shapes,decorations,fit,matrix,chains,scopes,positioning,arrows,automata,spy, shapes.callouts, patterns,calc,decorations.pathreplacing,decorations.markings, shadows}
% \usepgflibrary{shapes.callouts}
\usetikzlibrary{shapes.callouts}
\usetikzlibrary{decorations.text}
% \usepgflibrary{decorations.markings,fadings,shadings}
% \usepackage{verbatim}
\usetikzlibrary{arrows.meta}
\usetikzlibrary{shapes.gates.logic.US,shapes.geometric,quotes}
\usetikzlibrary{patterns}
\usetikzlibrary{fpu}

\usepackage{amsmath}
\usepackage{color}
\usepackage{ifthen}

\usepackage{hyperref}

\hypersetup{
    colorlinks=true,          % false: boxed links; true: colored links
    linkcolor=black,           % color of internal links
    citecolor=black,           % color of links to bibliography
}
\begin{document}

% @Author: Ismael Seidel
% @Date:   2020-02-07
% @Last Modified by:   Ismael Seidel
% @Last Modified time: 2020-02-07

\tikzset{
  multidocument/.style={
    shape=tape,
    draw,
    fill=white,
    tape bend top=none,
    double copy shadow},
}

\tikzset{
  document/.style={
    shape=tape,
    draw,
    fill=white,
    tape bend top=none},
}

\tikzstyle{myArrow} = [->, >=latex', thick]
\tikzstyle{codec} = [dashed, draw, rounded corners, thick, inner sep=3mm]

\begin{tikzpicture}[node distance=5mm]
	\node[] (name) {::get\_encoder(configuration)};
	\node[circle, radius=1cm, fill=black, below=-2mm of name] (init) {};

	\node[below=1cm of init, text width=4cm, align=center, draw] (getPart) {\textbf{switch} \\ configuration-$>$\\get\_jpeg\_pleno\_part()};

	\node[below=2cm of getPart, decision] (isPart2) {is Part 2?};

	

	\node[right=1cm of getPart, draw, text width=5cm, align=center] (getLFEncoder) {::get\_lightfield\_encoder\\(configuration)};

	\node[draw, below=1cm of getLFEncoder] (getType) {configuration-$>$get\_type()};



	\node[below=2.5cm of getType, decision] (isTransform) {is\\ transform \\mode?};

	\node[below=1cm of isTransform, draw, text width=5cm, align=center] (JPLM4DTransformModeLightFieldEncoder) {JPLM4DTransformMode\\LightFieldEncoder(configuration)};

	\node[rounded corners, draw, below=1cm of JPLM4DTransformModeLightFieldEncoder] (return) {\large \textbf{return encoder}};

	\node[text width=4.5cm, align=center, rounded corners, draw] (exception) at (isPart2 |- return) {\textbf{default}: throw \\ Not Implemented Exception};

	\draw[myArrow] (init) -- (getPart);
	\draw[myArrow] (getPart) -- (isPart2);
	\draw[myArrow] (isPart2) --node[pos=0.05, right]{no} (exception);
	\draw[myArrow] (isPart2) -| node[pos=0.1, above]{yes} ($(getLFEncoder.north west)+(-.5,.5)$) -| (getLFEncoder);
	\draw[myArrow] (getLFEncoder) -- (getType);
	\draw[myArrow] (getType) -- (isTransform);
	\draw[myArrow] (isTransform) --node[pos=0.15, right]{yes} (JPLM4DTransformModeLightFieldEncoder);
	\draw[myArrow] (isTransform) --node[pos=0.05, above]{no} (isTransform -| isPart2);
	\draw[myArrow] (JPLM4DTransformModeLightFieldEncoder) -- (return);


\end{tikzpicture}

\end{document}