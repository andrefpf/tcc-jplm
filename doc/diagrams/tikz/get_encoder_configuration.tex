%!TEX jobname = get_encoder_configuration
%!TEX output_directory = ./output/

% @Author: Ismael Seidel
% @Date:   2020-02-05
% @Last Modified by:   Ismael Seidel
% @Last Modified time: 2020-02-07


\documentclass[a4paper,landscape]{article}
\usepackage{pgf,tikz}
\usepackage{verbatim}
\usepackage[active,tightpage]{preview}
\PreviewEnvironment{tikzpicture}
\setlength\PreviewBorder{0pt}%
\usetikzlibrary{calc,arrows,fit, positioning, arrows.meta, fit}
\usetikzlibrary{shapes.gates.logic.US,shapes.geometric,quotes}

\usepackage{pgf}
\usepackage{tikz}
\usetikzlibrary{backgrounds, shapes,decorations,fit,matrix,chains,scopes,positioning,arrows,automata,spy, shapes.callouts, patterns,calc,decorations.pathreplacing,decorations.markings, shadows}
% \usepgflibrary{shapes.callouts}
\usetikzlibrary{shapes.callouts}
\usetikzlibrary{decorations.text}
% \usepgflibrary{decorations.markings,fadings,shadings}
% \usepackage{verbatim}
\usetikzlibrary{arrows.meta}
\usetikzlibrary{shapes.gates.logic.US,shapes.geometric,quotes}
\usetikzlibrary{patterns}
\usetikzlibrary{fpu}

\tikzset{
  treenode/.style = {shape=rectangle, rounded corners,
                     draw, anchor=center,
                     text width=5em, align=center,
                     top color=white, bottom color=blue!10,
                     inner sep=1ex},
  decision/.style = {treenode, diamond, inner sep=0pt, outer sep=-1pt},
  root/.style     = {treenode, font=\Large, bottom color=red!30},
  env/.style      = {treenode, font=\ttfamily\normalsize},
  finish/.style   = {root, bottom color=green!40},
  dummy/.style    = {circle,draw}
}

\usepackage{amsmath}
%\usepackage{rtl}
\usepackage{color}
% \usepackage[left=1cm,right=1cm]{geometry}
\usepackage{ifthen}

\usepackage{hyperref}

\hypersetup{
    colorlinks=true,          % false: boxed links; true: colored links
    linkcolor=black,           % color of internal links
    citecolor=black,           % color of links to bibliography
}
\begin{document}

\tikzstyle{myArrow} = [->, >=latex', thick]

\begin{tikzpicture}[node distance=5mm]
	\node[] (name) {get\_configuration(args)};
	\node[circle, radius=1cm, fill=black, below=-2mm of name] (init) {};

	\node[below=of init, draw] (JPLMEncoderConfiguration) {JPLMEncoderConfiguration(args)};
	\node[below=2cm of JPLMEncoderConfiguration, decision] (isHelp) {is help?};

	\node[left=2cm of isHelp, draw, text width=5cm, align=center] (JPLMEncoderConfigurationLightFieldHelp) {JPLMEncoderConfiguration\\LightField(args)};
	\node[below=1cm of JPLMEncoderConfigurationLightFieldHelp, draw, text width=5cm, align=center] (JPLMEncoderConfigurationLightField4DTransformModeHelp) {JPLMEncoderConfiguration\\LightField\\4DTransformMode(args)};

	\node[below=1cm of isHelp, text width=4cm, align=center, draw] (getPart) {\textbf{switch} \\ get\_jpeg\_pleno\_part()};

	\node[below=2cm of getPart, decision] (isPart2) {is Part 2?};

	\node[below=1cm of isPart2, text width=4.5cm, align=center, rounded corners, draw] (exception) {\textbf{default}: throw \\ Not Implemented Exception};

	\node[right=1cm of JPLMEncoderConfiguration, draw, text width=5cm, align=center] (JPLMEncoderConfigurationLightField) {JPLMEncoderConfiguration\\LightField(args)};

	\node[draw, below=1cm of JPLMEncoderConfigurationLightField] (getType) {get\_type()};



	\node[below=2cm of getType, decision] (isTransform) {is\\ transform \\mode?};
	
	\node[below=1cm of isTransform, draw, text width=5cm, align=center] (JPLMEncoderConfigurationLightField4DPredictionMode) {JPLMEncoderConfiguration\\LightField\\4DPredictionMode(args)};

	\node[right=1cm of JPLMEncoderConfigurationLightField4DPredictionMode, draw, text width=5cm, align=center] (JPLMEncoderConfigurationLightField4DTransformMode) {JPLMEncoderConfiguration\\LightField\\4DTransformMode(args)};

	

	\node[rounded corners, draw] (return) at (exception -| JPLMEncoderConfigurationLightField4DTransformMode) {\large \textbf{return configuration}};

	\node[rounded corners, draw] (exit) at (exception -| JPLMEncoderConfigurationLightField4DTransformModeHelp) {exit(0)};

	% \node[right=1cm of isPart2, text width=4cm, align=center, draw] (getPart) {\textbf{switch} \\ get\_type()};
	\draw[myArrow] (init) -- (JPLMEncoderConfiguration);
	\draw[myArrow] (JPLMEncoderConfiguration) -- (isHelp);
	\draw[myArrow] (isHelp) -- node[pos=0.1, above]{yes} (JPLMEncoderConfigurationLightFieldHelp);
	\draw[myArrow] (JPLMEncoderConfigurationLightFieldHelp) -- (JPLMEncoderConfigurationLightField4DTransformModeHelp);
	\draw[myArrow] (isHelp) -- node[pos=0.1, right]{no} (getPart);
	\draw[myArrow] (getPart) -- (isPart2);
	\draw[myArrow] (isPart2) --node[pos=0.1, right]{no} (exception);
	\draw[myArrow] (isPart2) -| node[pos=0.1, above]{yes} ($(JPLMEncoderConfigurationLightField.north west)+(-.5,.5)$) -| (JPLMEncoderConfigurationLightField);
	\draw[myArrow] (JPLMEncoderConfigurationLightField) -- (getType);
	\draw[myArrow] (getType) -- (isTransform);
	\draw[myArrow] (isTransform) -|node[pos=0.05, above]{yes} (JPLMEncoderConfigurationLightField4DTransformMode);
	\draw[myArrow] (isTransform) --node[pos=0.1, right]{no} (JPLMEncoderConfigurationLightField4DPredictionMode);
	\draw[myArrow] (JPLMEncoderConfigurationLightField4DPredictionMode) |- (exception);
	\draw[myArrow] (JPLMEncoderConfigurationLightField4DTransformModeHelp) -- (exit);
	\draw[myArrow] (JPLMEncoderConfigurationLightField4DTransformMode) -- (return);


\end{tikzpicture}

\end{document}