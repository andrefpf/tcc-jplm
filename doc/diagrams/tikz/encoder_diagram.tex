%!TEX jobname = encoder_diagram
%!TEX output_directory = ./output/

% @Author: Ismael Seidel
% @Date:   2020-02-05
% @Last Modified by:   Ismael Seidel
% @Last Modified time: 2020-02-07

\documentclass[a4paper,landscape]{article}
\usepackage{pgf,tikz}
\usepackage{verbatim}
\usepackage[active,tightpage]{preview}
\PreviewEnvironment{tikzpicture}
\setlength\PreviewBorder{0pt}%
\usetikzlibrary{calc,arrows,fit, positioning, arrows.meta, fit}
\usetikzlibrary{shapes.gates.logic.US,shapes.geometric,quotes}

\usepackage{pgf}
\usepackage{tikz}
\usetikzlibrary{backgrounds, shapes,decorations,fit,matrix,chains,scopes,positioning,arrows,automata,spy, shapes.callouts, patterns,calc,decorations.pathreplacing,decorations.markings, shadows}
% \usepgflibrary{shapes.callouts}
\usetikzlibrary{shapes.callouts}
\usetikzlibrary{decorations.text}
% \usepgflibrary{decorations.markings,fadings,shadings}
% \usepackage{verbatim}
\usetikzlibrary{arrows.meta}
\usetikzlibrary{shapes.gates.logic.US,shapes.geometric,quotes}
\usetikzlibrary{patterns}
\usetikzlibrary{fpu}

\usepackage{amsmath}
\usepackage{color}
\usepackage{ifthen}

\usepackage{hyperref}

\hypersetup{
    colorlinks=true,          % false: boxed links; true: colored links
    linkcolor=black,           % color of internal links
    citecolor=black,           % color of links to bibliography
}
\begin{document}



% @Author: Ismael Seidel
% @Date:   2020-02-07
% @Last Modified by:   Ismael Seidel
% @Last Modified time: 2020-02-07

\tikzset{
  multidocument/.style={
    shape=tape,
    draw,
    fill=white,
    tape bend top=none,
    double copy shadow},
}

\tikzset{
  document/.style={
    shape=tape,
    draw,
    fill=white,
    tape bend top=none},
}

\tikzstyle{myArrow} = [->, >=latex', thick]
\tikzstyle{codec} = [dashed, draw, rounded corners, thick, inner sep=3mm]

\begin{tikzpicture}[node distance=5mm]
	\node[] (args) {\texttt{./jpl-encoder ...args...}};
	\node[init, below=-1mm of args] (init) {};

	\node[draw, below=of init, fill=white, text width=3.5cm, align=center] (config) {configuration=\\ConfigurationFactory::\\get\_configuration(args)};
	\node[left=.5cm of config, multidocument, text width=1cm, align=center] (cfg) {*.cfg};

	\node[draw, below=of config, fill=white, text width=4.5cm, align=center] (encoder) {encoder=\\CodecFactory::\\get\_encoder(configuration)};
	\node[draw, below=of encoder, fill=white] (run) {encoder.run()};
	\node[left=1cm of run, multidocument, text width=1.5cm, align=center] (input) {plenoptic\\input};
	\node[draw, below=of run, fill=white, text width=3.5cm, align=center] (stream) {stream $\ll$ \\ encoder.get\_jpl\_file()};
	\node[draw, below=of stream, fill=white, rounded corners] (exit) {exit(0)};

	\node[right=.4cm of stream, document] (jplFile) {file.jpl};


	\begin{scope}[on background layer]
		\node[codec,fit=(config) (encoder) (run) (stream) (exit)] (co) {};
	\end{scope}	


	\draw[myArrow] (init) -- (config);
	\draw[myArrow] (cfg) -- (config);
	\draw[myArrow] (config) -- (encoder);
	\draw[myArrow] (encoder) -- (run);
	\draw[myArrow] (input) -- (run);	
	\draw[myArrow] (run) -- (stream);
	\draw[myArrow] (stream) -- (jplFile);
	\draw[myArrow] (stream) -- (exit);
\end{tikzpicture}

% The \gls{jplm} encoder execution can be divided in four main phases (Fig.~\ref{fig.encoder.phases}).
% The first one is the configuration parsing, that organizes the relevant information from the command line.
% The second one instantiates one specific encoder from the encoder hierarchy, according to the parsed configurations.
% The encoding itself occurs during the third phase, when the run method is called. 
% Finally, the fourth phase redirects the jpl file from the encoder to a given output stream, such as the filesystem.
% In this last step, all boxes are recursivelly redirected to the stream, according to JPEG Pleno box structure. 

\end{document}
